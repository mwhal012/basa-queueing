\documentclass[hidelinks,14pt]{extarticle} % use larger type; default would be 10pt

\usepackage[luatex]{hyperref}

%%% PAGE DIMENSIONS
\usepackage[
	verbose=true,
	driver=luatex,
	a4paper,
	margin=2cm,
	left=2.5cm,
	footskip=1.5cm,
	includeheadfoot
]{geometry}
\usepackage{fancyhdr}

\usepackage{graphicx} % support the \includegraphics command and options

\usepackage{booktabs,bookmark} % for much better looking tables
\usepackage{array} % for better arrays (eg matrices) in maths
\usepackage{subfig} % make it possible to include more than one captioned figure/table in a single float
% These packages are all incorporated in the memoir class to one degree or another...

\usepackage{amsmath}
\usepackage{fontspec}
\usepackage{unicode-math}

\AtBeginDocument{% to do this after unicode-math has done its work
  \renewcommand{\setminus}{\mathbin{\backslash}}%
}
\usepackage{enumitem}
	\setlist{left=-\parindent .. 0pt,labelsep=0.25\parindent,font=\bfseries}

\usepackage{datetime2}
\usepackage{statrep}

\usepackage{textcomp}

\usepackage{tabularx}
\usepackage{multirow}
\usepackage{bigstrut}
\usepackage{makecell}

\usepackage{float}

\usepackage{pdfpages}

%%% END Article customizations

\makeatletter
\let\save@mathaccent\mathaccent
\newcommand*\if@single[3]{%
  \setbox0\hbox{${\mathaccent"0362{#1}}^H$}%
  \setbox2\hbox{${\mathaccent"0362{\kern0pt#1}}^H$}%
  \ifdim\ht0=\ht2 #3\else #2\fi
  }
%The bar will be moved to the right by a half of \macc@kerna, which is computed by amsmath:
\newcommand*\rel@kern[1]{\kern#1\dimexpr\macc@kerna}
%If there's a superscript following the bar, then no negative kern may follow the bar;
%an additional {} makes sure that the superscript is high enough in this case:
\DeclareRobustCommand\widebar[1]{\@ifnextchar^{{\wide@bar{#1}{0}}}{\wide@bar{#1}{1}}}
%Use a separate algorithm for single symbols:
\newcommand*\wide@bar[2]{\if@single{#1}{\wide@bar@{#1}{#2}{1}}{\wide@bar@{#1}{#2}{2}}}
\newcommand*\wide@bar@[3]{%
  \begingroup
  \def\mathaccent##1##2{%
%Enable nesting of accents:
    \let\mathaccent\save@mathaccent
%If there's more than a single symbol, use the first character instead (see below):
    \if#32 \let\macc@nucleus\first@char \fi
%Determine the italic correction:
    \setbox\z@\hbox{$\macc@style{\macc@nucleus}_{}$}%
    \setbox\tw@\hbox{$\macc@style{\macc@nucleus}{}_{}$}%
    \dimen@\wd\tw@
    \advance\dimen@-\wd\z@
%Now \dimen@ is the italic correction of the symbol.
    \divide\dimen@ 3
    \@tempdima\wd\tw@
    \advance\@tempdima-\scriptspace
%Now \@tempdima is the width of the symbol.
    \divide\@tempdima 10
    \advance\dimen@-\@tempdima
%Now \dimen@ = (italic correction / 3) - (Breite / 10)
    \ifdim\dimen@>\z@ \dimen@0pt\fi
%The bar will be shortened in the case \dimen@<0 !
    \rel@kern{0.6}\kern-\dimen@
    \if#31
      \overline{\rel@kern{-0.6}\kern\dimen@\macc@nucleus\rel@kern{0.4}\kern\dimen@}%
      \advance\dimen@0.4\dimexpr\macc@kerna
%Place the combined final kern (-\dimen@) if it is >0 or if a superscript follows:
      \let\final@kern#2%
      \ifdim\dimen@<\z@ \let\final@kern1\fi
      \if\final@kern1 \kern-\dimen@\fi
    \else
      \overline{\rel@kern{-0.6}\kern\dimen@#1}%
    \fi
  }%
  \macc@depth\@ne
  \let\math@bgroup\@empty \let\math@egroup\macc@set@skewchar
  \mathsurround\z@ \frozen@everymath{\mathgroup\macc@group\relax}%
  \macc@set@skewchar\relax
  \let\mathaccentV\macc@nested@a
%The following initialises \macc@kerna and calls \mathaccent:
  \if#31
    \macc@nested@a\relax111{#1}%
  \else
%If the argument consists of more than one symbol, and if the first token is
%a letter, use that letter for the computations:
    \def\gobble@till@marker##1\endmarker{}%
    \futurelet\first@char\gobble@till@marker#1\endmarker
    \ifcat\noexpand\first@char A\else
      \def\first@char{}%
    \fi
    \macc@nested@a\relax111{\first@char}%
  \fi
  \endgroup
}
\makeatother

\newcommand{\tab}{\hspace*{\parindent}}
\newcommand{\cN}{\mathcal{N}}
\newcommand{\cF}{\mathcal{F}}
\newcommand{\cG}{\mathcal{G}}
\newcommand{\defs}{\text{Defs.}}
\newcommand{\Fc}{F\!_c}
\newcommand{\Fd}{F\!_d}
\newcommand{\dif}{\,d\!}
\newcommand{\dx}{\dif x}
\newcommand{\dF}{d\! F}
\newcommand{\given}{\,\vert\,}
\DeclareMathOperator{\argmin}{\text{argmin}}

\allowdisplaybreaks

\begin{document}

\[P[N(t+s) - N(s) = n] = e^{-\lambda t} \dfrac{(\lambda t)^n}{n!}\, , \quad n = 0, 1, \dots\]
\[p(x) = P(W_q\leq x) = 1 - \dfrac{\lambda}{\mu}e^{-(\mu-\lambda)x}\]
\[\overline{W}_q = \dfrac{\lambda}{\mu(\mu - \lambda)}\]
\[\widehat{\mu}_M = \dfrac{\overline{W}_q\lambda + \sqrt{(\overline{W}_q\lambda)^2 + 4\overline{W}_q\lambda}}{2 \overline{W}_q}\]
\[\widehat{p}_M(x) = 1 - \dfrac{\lambda}{\widehat{\mu}_M}e^{-(\widehat{\mu}_M - \lambda)x}\,, \quad 0\leq \widehat{p}_M(x) \leq 1\,\text{ if } \lambda< \widehat{\mu}_M\]
\[\left(\widehat{p}_M(x), x\right)\]
\[(p(x),x)\]
\[\widehat{\mu}_R = \widehat{a}c + \widehat{b}\lambda\]
\[\left(\dfrac{\lambda}{c},\dfrac{\widehat{\mu}_M}{c}\right)\]
\[\left(\widehat{p}_R(x),x\right)\]
\[\widehat{p}_R(x) = 1 - \dfrac{\lambda}{\widehat{a}c + \widehat{b}\lambda} e^{-\left(\widehat{a}c +\widehat{b}\lambda - \lambda\right)x}\,,\quad\lambda<\widehat{\mu}_R\]\pagebreak
\[0\leq p \leq 1\,, \quad0\leq\dfrac{\lambda}{\widehat{a}c + \widehat{b}\lambda}\leq1\]
\begin{align*}
	p &= 1 - \dfrac{\lambda}{\widehat{a}c + \widehat{b}\lambda} e^{-\left(\widehat{a}c +\widehat{b}\lambda - \lambda\right)x}\\
	\dfrac{\lambda}{\widehat{a}c + \widehat{b}\lambda} e^{-\left(\widehat{a}c +\widehat{b}\lambda - \lambda\right)x} &= 1 - p\\
	\dfrac{\lambda}{1-p}e^{\lambda x-\left(\widehat{a}c+\widehat{b}\lambda\right)x}&=\widehat{a}c+\widehat{b}\lambda\\
	\dfrac{\lambda}{1-p}e^{\lambda x} &= \left(\widehat{a}c+\widehat{b}\lambda\right)e^{\left(\widehat{a}c+\widehat{b}\lambda\right)x}\\
	\dfrac{\lambda x}{1-p}e^{\lambda x} &= \left(\widehat{a}c+\widehat{b}\lambda\right)xe^{\left(\widehat{a}c+\widehat{b}\lambda\right)x}\\
	W_0\left(\dfrac{\lambda}{1-p}e^{\lambda x}\right) &= \left(\widehat{a}c + \widehat{b}\lambda\right)x\\
	c_R :&= \dfrac{1}{\widehat{a}x}\left\{W_0\left(\dfrac{\lambda x}{1-p}e^{\lambda x}\right) - \widehat{b}\lambda x\right\}
\end{align*}
\[c\approx c_r\]

\end{document}